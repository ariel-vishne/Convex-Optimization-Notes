% Options for packages loaded elsewhere
\PassOptionsToPackage{unicode}{hyperref}
\PassOptionsToPackage{hyphens}{url}
%
\documentclass[
]{article}
\usepackage{amsmath,amssymb}
\usepackage{lmodern}
\usepackage{iftex}
\ifPDFTeX
  \usepackage[T1]{fontenc}
  \usepackage[utf8]{inputenc}
  \usepackage{textcomp} % provide euro and other symbols
\else % if luatex or xetex
  \usepackage{unicode-math}
  \defaultfontfeatures{Scale=MatchLowercase}
  \defaultfontfeatures[\rmfamily]{Ligatures=TeX,Scale=1}
\fi
% Use upquote if available, for straight quotes in verbatim environments
\IfFileExists{upquote.sty}{\usepackage{upquote}}{}
\IfFileExists{microtype.sty}{% use microtype if available
  \usepackage[]{microtype}
  \UseMicrotypeSet[protrusion]{basicmath} % disable protrusion for tt fonts
}{}
\makeatletter
\@ifundefined{KOMAClassName}{% if non-KOMA class
  \IfFileExists{parskip.sty}{%
    \usepackage{parskip}
  }{% else
    \setlength{\parindent}{0pt}
    \setlength{\parskip}{6pt plus 2pt minus 1pt}}
}{% if KOMA class
  \KOMAoptions{parskip=half}}
\makeatother
\usepackage{xcolor}
\IfFileExists{xurl.sty}{\usepackage{xurl}}{} % add URL line breaks if available
\IfFileExists{bookmark.sty}{\usepackage{bookmark}}{\usepackage{hyperref}}
\hypersetup{
  pdftitle={Nonlinear Optimization Ex.1},
  pdfauthor={Ariel Vishne 204149371},
  hidelinks,
  pdfcreator={LaTeX via pandoc}}
\urlstyle{same} % disable monospaced font for URLs
\usepackage[margin=1in]{geometry}
\usepackage{color}
\usepackage{fancyvrb}
\newcommand{\VerbBar}{|}
\newcommand{\VERB}{\Verb[commandchars=\\\{\}]}
\DefineVerbatimEnvironment{Highlighting}{Verbatim}{commandchars=\\\{\}}
% Add ',fontsize=\small' for more characters per line
\usepackage{framed}
\definecolor{shadecolor}{RGB}{248,248,248}
\newenvironment{Shaded}{\begin{snugshade}}{\end{snugshade}}
\newcommand{\AlertTok}[1]{\textcolor[rgb]{0.94,0.16,0.16}{#1}}
\newcommand{\AnnotationTok}[1]{\textcolor[rgb]{0.56,0.35,0.01}{\textbf{\textit{#1}}}}
\newcommand{\AttributeTok}[1]{\textcolor[rgb]{0.77,0.63,0.00}{#1}}
\newcommand{\BaseNTok}[1]{\textcolor[rgb]{0.00,0.00,0.81}{#1}}
\newcommand{\BuiltInTok}[1]{#1}
\newcommand{\CharTok}[1]{\textcolor[rgb]{0.31,0.60,0.02}{#1}}
\newcommand{\CommentTok}[1]{\textcolor[rgb]{0.56,0.35,0.01}{\textit{#1}}}
\newcommand{\CommentVarTok}[1]{\textcolor[rgb]{0.56,0.35,0.01}{\textbf{\textit{#1}}}}
\newcommand{\ConstantTok}[1]{\textcolor[rgb]{0.00,0.00,0.00}{#1}}
\newcommand{\ControlFlowTok}[1]{\textcolor[rgb]{0.13,0.29,0.53}{\textbf{#1}}}
\newcommand{\DataTypeTok}[1]{\textcolor[rgb]{0.13,0.29,0.53}{#1}}
\newcommand{\DecValTok}[1]{\textcolor[rgb]{0.00,0.00,0.81}{#1}}
\newcommand{\DocumentationTok}[1]{\textcolor[rgb]{0.56,0.35,0.01}{\textbf{\textit{#1}}}}
\newcommand{\ErrorTok}[1]{\textcolor[rgb]{0.64,0.00,0.00}{\textbf{#1}}}
\newcommand{\ExtensionTok}[1]{#1}
\newcommand{\FloatTok}[1]{\textcolor[rgb]{0.00,0.00,0.81}{#1}}
\newcommand{\FunctionTok}[1]{\textcolor[rgb]{0.00,0.00,0.00}{#1}}
\newcommand{\ImportTok}[1]{#1}
\newcommand{\InformationTok}[1]{\textcolor[rgb]{0.56,0.35,0.01}{\textbf{\textit{#1}}}}
\newcommand{\KeywordTok}[1]{\textcolor[rgb]{0.13,0.29,0.53}{\textbf{#1}}}
\newcommand{\NormalTok}[1]{#1}
\newcommand{\OperatorTok}[1]{\textcolor[rgb]{0.81,0.36,0.00}{\textbf{#1}}}
\newcommand{\OtherTok}[1]{\textcolor[rgb]{0.56,0.35,0.01}{#1}}
\newcommand{\PreprocessorTok}[1]{\textcolor[rgb]{0.56,0.35,0.01}{\textit{#1}}}
\newcommand{\RegionMarkerTok}[1]{#1}
\newcommand{\SpecialCharTok}[1]{\textcolor[rgb]{0.00,0.00,0.00}{#1}}
\newcommand{\SpecialStringTok}[1]{\textcolor[rgb]{0.31,0.60,0.02}{#1}}
\newcommand{\StringTok}[1]{\textcolor[rgb]{0.31,0.60,0.02}{#1}}
\newcommand{\VariableTok}[1]{\textcolor[rgb]{0.00,0.00,0.00}{#1}}
\newcommand{\VerbatimStringTok}[1]{\textcolor[rgb]{0.31,0.60,0.02}{#1}}
\newcommand{\WarningTok}[1]{\textcolor[rgb]{0.56,0.35,0.01}{\textbf{\textit{#1}}}}
\usepackage{graphicx}
\makeatletter
\def\maxwidth{\ifdim\Gin@nat@width>\linewidth\linewidth\else\Gin@nat@width\fi}
\def\maxheight{\ifdim\Gin@nat@height>\textheight\textheight\else\Gin@nat@height\fi}
\makeatother
% Scale images if necessary, so that they will not overflow the page
% margins by default, and it is still possible to overwrite the defaults
% using explicit options in \includegraphics[width, height, ...]{}
\setkeys{Gin}{width=\maxwidth,height=\maxheight,keepaspectratio}
% Set default figure placement to htbp
\makeatletter
\def\fps@figure{htbp}
\makeatother
\setlength{\emergencystretch}{3em} % prevent overfull lines
\providecommand{\tightlist}{%
  \setlength{\itemsep}{0pt}\setlength{\parskip}{0pt}}
\setcounter{secnumdepth}{-\maxdimen} % remove section numbering
\ifLuaTeX
  \usepackage{selnolig}  % disable illegal ligatures
\fi

\title{Nonlinear Optimization Ex.1}
\author{Ariel Vishne 204149371}
\date{02 05 2022}

\begin{document}
\maketitle

\hypertarget{ex.-1}{%
\subsection{Ex. 1}\label{ex.-1}}

We require:

\[
f(x,y) = 100(y-x^2)^2+(1-x)^2 = 100(y^2-2x^2y+x^4) + 1-2x+x^2= 100x^4+x^2-2x-200x^2y+100y^2+1
\] Defined as a function for later use in code we have:

\begin{Shaded}
\begin{Highlighting}[]
\NormalTok{f }\OtherTok{\textless{}{-}} \ControlFlowTok{function}\NormalTok{(x,y)\{}
 \FunctionTok{return}\NormalTok{( }\DecValTok{100} \SpecialCharTok{*}\NormalTok{ ((y }\SpecialCharTok{{-}}\NormalTok{ x }\SpecialCharTok{\^{}} \DecValTok{2}\NormalTok{) }\SpecialCharTok{\^{}} \DecValTok{2}\NormalTok{) }\SpecialCharTok{+}\NormalTok{ ((}\DecValTok{1} \SpecialCharTok{{-}}\NormalTok{ x) }\SpecialCharTok{\^{}} \DecValTok{2}\NormalTok{)) }
\NormalTok{\}}
\end{Highlighting}
\end{Shaded}

The gradient is: \[
\nabla f(x,y)= \begin{bmatrix} \frac{\partial f(x,y)}{\partial x} \\ \frac{\partial f(x,y)}{\partial y} \end{bmatrix} = \begin{bmatrix} 400x^3+2x-2-400xy \\ -200x^2+200y \end{bmatrix}
\]

Therefore the formula for gradient descent, for given \(Z^{n}=(x,y)\)
will be defined as: \[
Z^{n+1} = Z^{n} - \alpha \bullet \nabla f(Z^{n})=\begin{bmatrix} x \\ y \end{bmatrix} - \alpha \bullet \begin{bmatrix} 400x^3+2x-2-400xy \\ -200x^2+200y \end{bmatrix}=\begin{bmatrix} x - \alpha\bullet (400x^3+2x-2-400xy) \\ y-\alpha \bullet (-200x^2+200y) \end{bmatrix}\\ =\begin{bmatrix} x - 2\alpha\bullet (200x^3+x-1-200xy) \\ y-200\alpha \bullet (-x^2+y) \end{bmatrix}
\] We will also create the gradient descent function to be used later in
the code. This function recieves given \(Z^{n} = (x,y)\) and \(\alpha\)
and computes the next iteration.

\begin{Shaded}
\begin{Highlighting}[]
\NormalTok{compute.gradient.f }\OtherTok{\textless{}{-}} \ControlFlowTok{function}\NormalTok{(x,y)\{}
  \FunctionTok{return}\NormalTok{(list}
\NormalTok{         (}
\NormalTok{           (}\DecValTok{2} \SpecialCharTok{*}\NormalTok{ ((}\DecValTok{200} \SpecialCharTok{*}\NormalTok{ (x}\SpecialCharTok{\^{}}\DecValTok{3}\NormalTok{)) }\SpecialCharTok{+}\NormalTok{  x }\SpecialCharTok{{-}} \DecValTok{1} \SpecialCharTok{{-}}\NormalTok{ (}\DecValTok{200} \SpecialCharTok{*}\NormalTok{ y }\SpecialCharTok{*}\NormalTok{ x))),}
           \DecValTok{200} \SpecialCharTok{*}\NormalTok{ ((}\SpecialCharTok{{-}}\DecValTok{1} \SpecialCharTok{*}\NormalTok{ (x }\SpecialCharTok{\^{}} \DecValTok{2}\NormalTok{) }\SpecialCharTok{+}\NormalTok{ y))}
\NormalTok{              )}
\NormalTok{         )}
\NormalTok{\}}

\NormalTok{next.iteration }\OtherTok{\textless{}{-}} \ControlFlowTok{function}\NormalTok{(x,y,alpha)\{}
\NormalTok{  gradient }\OtherTok{\textless{}{-}} \FunctionTok{compute.gradient.f}\NormalTok{(x,y)}
\NormalTok{  nabla.f}\FloatTok{.1} \OtherTok{\textless{}{-}}\NormalTok{ gradient[[}\DecValTok{1}\NormalTok{]]}
\NormalTok{  nabla.f}\FloatTok{.2} \OtherTok{\textless{}{-}}\NormalTok{ gradient[[}\DecValTok{2}\NormalTok{]]}
  \FunctionTok{return}\NormalTok{(list}
\NormalTok{            (}
\NormalTok{              x }\SpecialCharTok{{-}}\NormalTok{ (alpha }\SpecialCharTok{*}\NormalTok{ nabla.f}\FloatTok{.1}\NormalTok{),}
\NormalTok{              y }\SpecialCharTok{{-}}\NormalTok{ (alpha }\SpecialCharTok{*}\NormalTok{ nabla.f}\FloatTok{.2}\NormalTok{)}
\NormalTok{              )}
\NormalTok{         )}
\NormalTok{\}}
\end{Highlighting}
\end{Shaded}

We need to find the optimal \(\alpha\) for this procedure. We will
define a new function: \[
g(\alpha) = f(Z^{n} - \alpha \bullet \nabla f(Z^{n}))
\] We will use golden search.

\begin{Shaded}
\begin{Highlighting}[]
\NormalTok{g }\OtherTok{\textless{}{-}} \ControlFlowTok{function}\NormalTok{(x,y,alpha)\{}
\NormalTok{  res }\OtherTok{\textless{}{-}} \FunctionTok{next.iteration}\NormalTok{(x,y,alpha)}
\NormalTok{  x }\OtherTok{\textless{}{-}}\NormalTok{ res[[}\DecValTok{1}\NormalTok{]]}
\NormalTok{  y }\OtherTok{\textless{}{-}}\NormalTok{ res[[}\DecValTok{2}\NormalTok{]]}
  \FunctionTok{return}\NormalTok{(}\FunctionTok{f}\NormalTok{(x,y))}
\NormalTok{\}}

\NormalTok{golden.ratio }\OtherTok{\textless{}{-}}\NormalTok{ ((}\DecValTok{1} \SpecialCharTok{+} \FunctionTok{sqrt}\NormalTok{(}\DecValTok{5}\NormalTok{)) }\SpecialCharTok{/} \DecValTok{2}\NormalTok{)}

\NormalTok{compute.alpha }\OtherTok{\textless{}{-}} \ControlFlowTok{function}\NormalTok{(a, b,x,y, tol)\{}
\NormalTok{    d }\OtherTok{\textless{}{-}}\NormalTok{ b }\SpecialCharTok{{-}}\NormalTok{ ((b}\SpecialCharTok{{-}}\NormalTok{a) }\SpecialCharTok{/}\NormalTok{ golden.ratio)}
\NormalTok{    c }\OtherTok{\textless{}{-}}\NormalTok{ a }\SpecialCharTok{+}\NormalTok{ ((b}\SpecialCharTok{{-}}\NormalTok{a) }\SpecialCharTok{/}\NormalTok{ golden.ratio)}
    
    \ControlFlowTok{while}\NormalTok{(}\FunctionTok{abs}\NormalTok{(d }\SpecialCharTok{{-}}\NormalTok{ c) }\SpecialCharTok{\textgreater{}}\NormalTok{ tol)\{}
      \ControlFlowTok{if}\NormalTok{ (}\FunctionTok{g}\NormalTok{(x,y,d) }\SpecialCharTok{\textless{}} \FunctionTok{g}\NormalTok{(x,y,c))\{}
\NormalTok{        b }\OtherTok{\textless{}{-}}\NormalTok{ c}
\NormalTok{      \}}
      \ControlFlowTok{else}\NormalTok{\{}
\NormalTok{        a }\OtherTok{\textless{}{-}}\NormalTok{ d}
\NormalTok{      \}}
\NormalTok{      d }\OtherTok{\textless{}{-}}\NormalTok{ b }\SpecialCharTok{{-}}\NormalTok{ ((b}\SpecialCharTok{{-}}\NormalTok{a) }\SpecialCharTok{/}\NormalTok{ golden.ratio)}
\NormalTok{      c }\OtherTok{\textless{}{-}}\NormalTok{ a }\SpecialCharTok{+}\NormalTok{ ((b}\SpecialCharTok{{-}}\NormalTok{a) }\SpecialCharTok{/}\NormalTok{ golden.ratio)}
\NormalTok{    \}}
    \FunctionTok{return}\NormalTok{((a}\SpecialCharTok{+}\NormalTok{b) }\SpecialCharTok{/} \DecValTok{2}\NormalTok{)}
\NormalTok{\}}
\end{Highlighting}
\end{Shaded}

Implementation of gradient descent method is therefore as follows.
Notice that at each iteration we use the optimal alpha through the
golden search.

\begin{Shaded}
\begin{Highlighting}[]
\NormalTok{sdescent }\OtherTok{\textless{}{-}} \ControlFlowTok{function}\NormalTok{(a0, tol, maxiter)\{}
  
\NormalTok{  x0 }\OtherTok{\textless{}{-}}\NormalTok{ a0[}\DecValTok{1}\NormalTok{]}
\NormalTok{  x }\OtherTok{\textless{}{-}}\NormalTok{ x0}
\NormalTok{  y0 }\OtherTok{\textless{}{-}}\NormalTok{ a0[}\DecValTok{2}\NormalTok{]}
\NormalTok{  y }\OtherTok{\textless{}{-}}\NormalTok{ y0}
\NormalTok{  a.t }\OtherTok{\textless{}{-}}\NormalTok{ a0}
\NormalTok{  a.values }\OtherTok{\textless{}{-}} \FunctionTok{matrix}\NormalTok{(}\DecValTok{0}\NormalTok{,}\AttributeTok{ncol =} \DecValTok{2}\NormalTok{, }\AttributeTok{nrow =}\NormalTok{ maxiter, }\AttributeTok{byrow =} \ConstantTok{TRUE}\NormalTok{)}
  
\NormalTok{  num.iter }\OtherTok{\textless{}{-}} \DecValTok{1}

  \ControlFlowTok{while}\NormalTok{(num.iter }\SpecialCharTok{\textless{}}\NormalTok{ maxiter)\{}
    
    \CommentTok{\#print("iteration number")}
    \CommentTok{\#print(num.iter)}
    \CommentTok{\#print("x, y values")}
    \CommentTok{\#print(round(x,6))}
    \CommentTok{\#print(round(y,6))}
    \CommentTok{\#print("function value at points")}
    \CommentTok{\#print(round(f(x,y), 6))}
    \CommentTok{\#print("=================")}
    
\NormalTok{    a.values[num.iter,] }\OtherTok{\textless{}{-}} \FunctionTok{c}\NormalTok{(x,y)}

\NormalTok{    num.iter }\OtherTok{\textless{}{-}}\NormalTok{ num.iter }\SpecialCharTok{+} \DecValTok{1}
    
    \CommentTok{\#print("Computing alpha")}
\NormalTok{    grad }\OtherTok{\textless{}{-}} \FunctionTok{compute.gradient.f}\NormalTok{(x,y)}
\NormalTok{    alpha }\OtherTok{\textless{}{-}} \FunctionTok{compute.alpha}\NormalTok{(}\DecValTok{0}\NormalTok{,}\DecValTok{1}\NormalTok{,x,y,tol)}
    \CommentTok{\#print("alpha value is")}
    \CommentTok{\#print(alpha)}
    \CommentTok{\#print("=================")}
\NormalTok{    a.t.minus.one }\OtherTok{\textless{}{-}}\NormalTok{ a.t}
    
\NormalTok{    a.t }\OtherTok{\textless{}{-}} \FunctionTok{next.iteration}\NormalTok{(x,y,alpha)}
    
    
\NormalTok{    x }\OtherTok{\textless{}{-}}\NormalTok{ a.t[[}\DecValTok{1}\NormalTok{]]}
\NormalTok{    y }\OtherTok{\textless{}{-}}\NormalTok{ a.t[[}\DecValTok{2}\NormalTok{]]}
    
    
    \ControlFlowTok{if}\NormalTok{ (}\FunctionTok{norm}\NormalTok{(}\FunctionTok{as.numeric}\NormalTok{(}\FunctionTok{c}\NormalTok{(x,y)) }\SpecialCharTok{{-}} \FunctionTok{as.numeric}\NormalTok{(a.t.minus.one), }\AttributeTok{type =} \StringTok{"2"}\NormalTok{) }\SpecialCharTok{\textless{}}\NormalTok{ tol)\{}
      \ControlFlowTok{break}
\NormalTok{    \}}
    
\NormalTok{  \}}
  \FunctionTok{return}\NormalTok{(}\FunctionTok{list}\NormalTok{(}
\NormalTok{    a.values,}
\NormalTok{    num.iter}
\NormalTok{  ))}
  
\NormalTok{\}}
\end{Highlighting}
\end{Shaded}

What is left is to produce the graph with the results. We show the
results with different resolutions.

\hypertarget{b-plotting-results}{%
\subsubsection{1b Plotting Results}\label{b-plotting-results}}

\hypertarget{first-setting---good-initialization}{%
\paragraph{First setting - Good
initialization}\label{first-setting---good-initialization}}

Our inital values are \(a^{0}=(0.95, 0.95)\). maxiter is \(1000\).

\begin{Shaded}
\begin{Highlighting}[]
\NormalTok{a0 }\OtherTok{\textless{}{-}} \FunctionTok{c}\NormalTok{(}\FloatTok{0.95}\NormalTok{, }\FloatTok{0.95}\NormalTok{)}
\NormalTok{tol }\OtherTok{\textless{}{-}} \DecValTok{10}\SpecialCharTok{\^{}{-}}\DecValTok{5}
\NormalTok{maxiter }\OtherTok{\textless{}{-}} \DecValTok{1000}

\NormalTok{res }\OtherTok{\textless{}{-}} \FunctionTok{sdescent}\NormalTok{(a0, tol, maxiter)}
\NormalTok{values }\OtherTok{\textless{}{-}}\NormalTok{ res[[}\DecValTok{1}\NormalTok{]]}
\NormalTok{values[maxiter,] }\OtherTok{\textless{}{-}} \ConstantTok{NA}
\NormalTok{iters }\OtherTok{\textless{}{-}}\NormalTok{ res[[}\DecValTok{2}\NormalTok{]]}
\FunctionTok{print}\NormalTok{(}\FunctionTok{paste}\NormalTok{(}\StringTok{"Iterations required:"}\NormalTok{,iters))}
\end{Highlighting}
\end{Shaded}

\begin{verbatim}
## [1] "Iterations required: 1000"
\end{verbatim}

\begin{Shaded}
\begin{Highlighting}[]
\CommentTok{\# redefine f so to work with ContourFunctions library}
\NormalTok{f2 }\OtherTok{\textless{}{-}} \ControlFlowTok{function}\NormalTok{(r)\{}
 \FunctionTok{return}\NormalTok{( }\DecValTok{100} \SpecialCharTok{*}\NormalTok{ ((r[}\DecValTok{2}\NormalTok{] }\SpecialCharTok{{-}}\NormalTok{ r[}\DecValTok{1}\NormalTok{] }\SpecialCharTok{\^{}} \DecValTok{2}\NormalTok{) }\SpecialCharTok{\^{}} \DecValTok{2}\NormalTok{) }\SpecialCharTok{+}\NormalTok{ ((}\DecValTok{1} \SpecialCharTok{{-}}\NormalTok{ r[}\DecValTok{1}\NormalTok{]) }\SpecialCharTok{\^{}} \DecValTok{2}\NormalTok{)) }
\NormalTok{\}}

\FunctionTok{print}\NormalTok{(}\StringTok{"ZOOM OUT values {-}10 to 10"}\NormalTok{)}
\end{Highlighting}
\end{Shaded}

\begin{verbatim}
## [1] "ZOOM OUT values -10 to 10"
\end{verbatim}

\begin{Shaded}
\begin{Highlighting}[]
\FunctionTok{cf\_func}\NormalTok{(f2, }\AttributeTok{xlim =} \FunctionTok{c}\NormalTok{(}\SpecialCharTok{{-}}\DecValTok{10}\NormalTok{, }\DecValTok{10}\NormalTok{), }\AttributeTok{ylim =} \FunctionTok{c}\NormalTok{(}\SpecialCharTok{{-}}\DecValTok{10}\NormalTok{, }\DecValTok{10}\NormalTok{), }\AttributeTok{bar =} \ConstantTok{TRUE}\NormalTok{, }\AttributeTok{pts =}\NormalTok{ values)}
\end{Highlighting}
\end{Shaded}

\includegraphics{ex2_files/figure-latex/graph1-gradient-1.pdf}

\begin{Shaded}
\begin{Highlighting}[]
\FunctionTok{print}\NormalTok{(}\StringTok{"ZOOM OUT values {-}2.5 to 2.5"}\NormalTok{)}
\end{Highlighting}
\end{Shaded}

\begin{verbatim}
## [1] "ZOOM OUT values -2.5 to 2.5"
\end{verbatim}

\begin{Shaded}
\begin{Highlighting}[]
\FunctionTok{cf\_func}\NormalTok{(f2, }\AttributeTok{xlim =} \FunctionTok{c}\NormalTok{(}\SpecialCharTok{{-}}\FloatTok{2.5}\NormalTok{, }\FloatTok{2.5}\NormalTok{), }\AttributeTok{ylim =} \FunctionTok{c}\NormalTok{(}\SpecialCharTok{{-}}\FloatTok{2.5}\NormalTok{, }\FloatTok{2.5}\NormalTok{), }\AttributeTok{bar =} \ConstantTok{TRUE}\NormalTok{, }\AttributeTok{pts =}\NormalTok{ values)}
\end{Highlighting}
\end{Shaded}

\includegraphics{ex2_files/figure-latex/graph1-gradient-2.pdf}

\begin{Shaded}
\begin{Highlighting}[]
\FunctionTok{print}\NormalTok{(}\StringTok{"ZOOM IN values 0.8{-}1.2"}\NormalTok{)}
\end{Highlighting}
\end{Shaded}

\begin{verbatim}
## [1] "ZOOM IN values 0.8-1.2"
\end{verbatim}

\begin{Shaded}
\begin{Highlighting}[]
\FunctionTok{cf\_func}\NormalTok{(f2, }\AttributeTok{xlim =} \FunctionTok{c}\NormalTok{(}\FloatTok{0.8}\NormalTok{, }\FloatTok{1.2}\NormalTok{), }\AttributeTok{ylim =} \FunctionTok{c}\NormalTok{(}\FloatTok{0.8}\NormalTok{, }\FloatTok{1.2}\NormalTok{), }\AttributeTok{bar =} \ConstantTok{TRUE}\NormalTok{, }\AttributeTok{pts =}\NormalTok{ values)}
\end{Highlighting}
\end{Shaded}

\includegraphics{ex2_files/figure-latex/graph1-gradient-3.pdf}

\begin{Shaded}
\begin{Highlighting}[]
\FunctionTok{print}\NormalTok{(}\StringTok{"ZOOM More IN values 0.975{-}1.025"}\NormalTok{)}
\end{Highlighting}
\end{Shaded}

\begin{verbatim}
## [1] "ZOOM More IN values 0.975-1.025"
\end{verbatim}

\begin{Shaded}
\begin{Highlighting}[]
\FunctionTok{cf\_func}\NormalTok{(f2, }\AttributeTok{xlim =} \FunctionTok{c}\NormalTok{(}\FloatTok{0.975}\NormalTok{, }\FloatTok{1.025}\NormalTok{), }\AttributeTok{ylim =} \FunctionTok{c}\NormalTok{(}\FloatTok{0.975}\NormalTok{, }\FloatTok{1.025}\NormalTok{), }\AttributeTok{bar =} \ConstantTok{TRUE}\NormalTok{, }\AttributeTok{pts =}\NormalTok{ values)}
\end{Highlighting}
\end{Shaded}

\includegraphics{ex2_files/figure-latex/graph1-gradient-4.pdf}

\hypertarget{second-setting---average-initialization}{%
\paragraph{Second setting - average
initialization}\label{second-setting---average-initialization}}

Our inital values are \(a^{0}=(5, 5)\). maxiter is \(10000\).

\begin{Shaded}
\begin{Highlighting}[]
\NormalTok{a0 }\OtherTok{\textless{}{-}} \FunctionTok{c}\NormalTok{(}\DecValTok{5}\NormalTok{,}\DecValTok{5}\NormalTok{)}
\NormalTok{tol }\OtherTok{\textless{}{-}} \DecValTok{10}\SpecialCharTok{\^{}{-}}\DecValTok{5}
\NormalTok{maxiter }\OtherTok{\textless{}{-}} \DecValTok{10000}

\NormalTok{res }\OtherTok{\textless{}{-}} \FunctionTok{sdescent}\NormalTok{(a0, tol, maxiter)}
\NormalTok{values }\OtherTok{\textless{}{-}}\NormalTok{ res[[}\DecValTok{1}\NormalTok{]]}
\NormalTok{values[maxiter,] }\OtherTok{\textless{}{-}} \ConstantTok{NA}
\NormalTok{iters }\OtherTok{\textless{}{-}}\NormalTok{ res[[}\DecValTok{2}\NormalTok{]]}
\FunctionTok{print}\NormalTok{(}\FunctionTok{paste}\NormalTok{(}\StringTok{"Iterations required:"}\NormalTok{,iters))}
\end{Highlighting}
\end{Shaded}

\begin{verbatim}
## [1] "Iterations required: 9418"
\end{verbatim}

\begin{Shaded}
\begin{Highlighting}[]
\FunctionTok{print}\NormalTok{(}\StringTok{"ZOOM OUT values {-}10 to 10"}\NormalTok{)}
\end{Highlighting}
\end{Shaded}

\begin{verbatim}
## [1] "ZOOM OUT values -10 to 10"
\end{verbatim}

\begin{Shaded}
\begin{Highlighting}[]
\FunctionTok{cf\_func}\NormalTok{(f2, }\AttributeTok{xlim =} \FunctionTok{c}\NormalTok{(}\SpecialCharTok{{-}}\DecValTok{10}\NormalTok{, }\DecValTok{10}\NormalTok{), }\AttributeTok{ylim =} \FunctionTok{c}\NormalTok{(}\SpecialCharTok{{-}}\DecValTok{10}\NormalTok{, }\DecValTok{10}\NormalTok{), }\AttributeTok{bar =} \ConstantTok{TRUE}\NormalTok{, }\AttributeTok{pts =}\NormalTok{ values)}
\end{Highlighting}
\end{Shaded}

\includegraphics{ex2_files/figure-latex/graph2-gradient-1.pdf}

\begin{Shaded}
\begin{Highlighting}[]
\FunctionTok{print}\NormalTok{(}\StringTok{"ZOOM OUT values {-}2.5 to 2.5"}\NormalTok{)}
\end{Highlighting}
\end{Shaded}

\begin{verbatim}
## [1] "ZOOM OUT values -2.5 to 2.5"
\end{verbatim}

\begin{Shaded}
\begin{Highlighting}[]
\FunctionTok{cf\_func}\NormalTok{(f2, }\AttributeTok{xlim =} \FunctionTok{c}\NormalTok{(}\SpecialCharTok{{-}}\FloatTok{2.5}\NormalTok{, }\FloatTok{2.5}\NormalTok{), }\AttributeTok{ylim =} \FunctionTok{c}\NormalTok{(}\SpecialCharTok{{-}}\FloatTok{2.5}\NormalTok{, }\FloatTok{2.5}\NormalTok{), }\AttributeTok{bar =} \ConstantTok{TRUE}\NormalTok{, }\AttributeTok{pts =}\NormalTok{ values)}
\end{Highlighting}
\end{Shaded}

\includegraphics{ex2_files/figure-latex/graph2-gradient-2.pdf}

\begin{Shaded}
\begin{Highlighting}[]
\FunctionTok{print}\NormalTok{(}\StringTok{"ZOOM IN values 0.8{-}1.2"}\NormalTok{)}
\end{Highlighting}
\end{Shaded}

\begin{verbatim}
## [1] "ZOOM IN values 0.8-1.2"
\end{verbatim}

\begin{Shaded}
\begin{Highlighting}[]
\FunctionTok{cf\_func}\NormalTok{(f2, }\AttributeTok{xlim =} \FunctionTok{c}\NormalTok{(}\FloatTok{0.8}\NormalTok{, }\FloatTok{1.2}\NormalTok{), }\AttributeTok{ylim =} \FunctionTok{c}\NormalTok{(}\FloatTok{0.8}\NormalTok{, }\FloatTok{1.2}\NormalTok{), }\AttributeTok{bar =} \ConstantTok{TRUE}\NormalTok{, }\AttributeTok{pts =}\NormalTok{ values)}
\end{Highlighting}
\end{Shaded}

\includegraphics{ex2_files/figure-latex/graph2-gradient-3.pdf}

\begin{Shaded}
\begin{Highlighting}[]
\FunctionTok{print}\NormalTok{(}\StringTok{"ZOOM More IN values 0.975{-}1.025"}\NormalTok{)}
\end{Highlighting}
\end{Shaded}

\begin{verbatim}
## [1] "ZOOM More IN values 0.975-1.025"
\end{verbatim}

\begin{Shaded}
\begin{Highlighting}[]
\FunctionTok{cf\_func}\NormalTok{(f2, }\AttributeTok{xlim =} \FunctionTok{c}\NormalTok{(}\FloatTok{0.975}\NormalTok{, }\FloatTok{1.025}\NormalTok{), }\AttributeTok{ylim =} \FunctionTok{c}\NormalTok{(}\FloatTok{0.975}\NormalTok{, }\FloatTok{1.025}\NormalTok{), }\AttributeTok{bar =} \ConstantTok{TRUE}\NormalTok{, }\AttributeTok{pts =}\NormalTok{ values)}
\end{Highlighting}
\end{Shaded}

\includegraphics{ex2_files/figure-latex/graph2-gradient-4.pdf}

\hypertarget{third-setting---bad-initialization}{%
\paragraph{Third setting - bad
initialization}\label{third-setting---bad-initialization}}

Our inital values are \(a^{0}=(-10, 10)\). maxiter is \(10000\).

\begin{Shaded}
\begin{Highlighting}[]
\NormalTok{a0 }\OtherTok{\textless{}{-}} \FunctionTok{c}\NormalTok{(}\SpecialCharTok{{-}}\DecValTok{10}\NormalTok{,}\DecValTok{10}\NormalTok{)}
\NormalTok{tol }\OtherTok{\textless{}{-}} \DecValTok{10}\SpecialCharTok{\^{}{-}}\DecValTok{5}
\NormalTok{maxiter }\OtherTok{\textless{}{-}} \DecValTok{10000}

\NormalTok{res }\OtherTok{\textless{}{-}} \FunctionTok{sdescent}\NormalTok{(a0, tol, maxiter)}
\NormalTok{values }\OtherTok{\textless{}{-}}\NormalTok{ res[[}\DecValTok{1}\NormalTok{]]}
\NormalTok{values[maxiter,] }\OtherTok{\textless{}{-}} \ConstantTok{NA}
\NormalTok{iters }\OtherTok{\textless{}{-}}\NormalTok{ res[[}\DecValTok{2}\NormalTok{]]}
\FunctionTok{print}\NormalTok{(}\FunctionTok{paste}\NormalTok{(}\StringTok{"Iterations required:"}\NormalTok{,iters))}
\end{Highlighting}
\end{Shaded}

\begin{verbatim}
## [1] "Iterations required: 10000"
\end{verbatim}

\begin{Shaded}
\begin{Highlighting}[]
\FunctionTok{print}\NormalTok{(}\StringTok{"ZOOM OUT values {-}10 to 10"}\NormalTok{)}
\end{Highlighting}
\end{Shaded}

\begin{verbatim}
## [1] "ZOOM OUT values -10 to 10"
\end{verbatim}

\begin{Shaded}
\begin{Highlighting}[]
\FunctionTok{cf\_func}\NormalTok{(f2, }\AttributeTok{xlim =} \FunctionTok{c}\NormalTok{(}\SpecialCharTok{{-}}\DecValTok{10}\NormalTok{, }\DecValTok{10}\NormalTok{), }\AttributeTok{ylim =} \FunctionTok{c}\NormalTok{(}\SpecialCharTok{{-}}\DecValTok{10}\NormalTok{, }\DecValTok{10}\NormalTok{), }\AttributeTok{bar =} \ConstantTok{TRUE}\NormalTok{, }\AttributeTok{pts =}\NormalTok{ values)}
\end{Highlighting}
\end{Shaded}

\includegraphics{ex2_files/figure-latex/graph3-gradient-1.pdf}

\begin{Shaded}
\begin{Highlighting}[]
\FunctionTok{print}\NormalTok{(}\StringTok{"ZOOM OUT values {-}2.5 to 2.5"}\NormalTok{)}
\end{Highlighting}
\end{Shaded}

\begin{verbatim}
## [1] "ZOOM OUT values -2.5 to 2.5"
\end{verbatim}

\begin{Shaded}
\begin{Highlighting}[]
\FunctionTok{cf\_func}\NormalTok{(f2, }\AttributeTok{xlim =} \FunctionTok{c}\NormalTok{(}\SpecialCharTok{{-}}\FloatTok{2.5}\NormalTok{, }\FloatTok{2.5}\NormalTok{), }\AttributeTok{ylim =} \FunctionTok{c}\NormalTok{(}\SpecialCharTok{{-}}\FloatTok{2.5}\NormalTok{, }\FloatTok{2.5}\NormalTok{), }\AttributeTok{bar =} \ConstantTok{TRUE}\NormalTok{, }\AttributeTok{pts =}\NormalTok{ values)}
\end{Highlighting}
\end{Shaded}

\includegraphics{ex2_files/figure-latex/graph3-gradient-2.pdf}

\begin{Shaded}
\begin{Highlighting}[]
\FunctionTok{print}\NormalTok{(}\StringTok{"ZOOM IN values 0.8{-}1.2"}\NormalTok{)}
\end{Highlighting}
\end{Shaded}

\begin{verbatim}
## [1] "ZOOM IN values 0.8-1.2"
\end{verbatim}

\begin{Shaded}
\begin{Highlighting}[]
\FunctionTok{cf\_func}\NormalTok{(f2, }\AttributeTok{xlim =} \FunctionTok{c}\NormalTok{(}\FloatTok{0.8}\NormalTok{, }\FloatTok{1.2}\NormalTok{), }\AttributeTok{ylim =} \FunctionTok{c}\NormalTok{(}\FloatTok{0.8}\NormalTok{, }\FloatTok{1.2}\NormalTok{), }\AttributeTok{bar =} \ConstantTok{TRUE}\NormalTok{, }\AttributeTok{pts =}\NormalTok{ values)}
\end{Highlighting}
\end{Shaded}

\includegraphics{ex2_files/figure-latex/graph3-gradient-3.pdf}

\begin{Shaded}
\begin{Highlighting}[]
\FunctionTok{print}\NormalTok{(}\StringTok{"ZOOM More IN values 0.975{-}1.025"}\NormalTok{)}
\end{Highlighting}
\end{Shaded}

\begin{verbatim}
## [1] "ZOOM More IN values 0.975-1.025"
\end{verbatim}

\begin{Shaded}
\begin{Highlighting}[]
\FunctionTok{cf\_func}\NormalTok{(f2, }\AttributeTok{xlim =} \FunctionTok{c}\NormalTok{(}\FloatTok{0.975}\NormalTok{, }\FloatTok{1.025}\NormalTok{), }\AttributeTok{ylim =} \FunctionTok{c}\NormalTok{(}\FloatTok{0.975}\NormalTok{, }\FloatTok{1.025}\NormalTok{), }\AttributeTok{bar =} \ConstantTok{TRUE}\NormalTok{, }\AttributeTok{pts =}\NormalTok{ values)}
\end{Highlighting}
\end{Shaded}

\includegraphics{ex2_files/figure-latex/graph3-gradient-4.pdf}

\hypertarget{c-speed-of-convergence}{%
\subsubsection{1c Speed of Convergence}\label{c-speed-of-convergence}}

We can see that the convergence is very slow, it is linear in relation
with number of iterations. We can see that when close to the point of
convergence there is no more `jumps' in the function, and the
convergence is very gradual. For the average initialization it took
about 9500 iterations to converge. On the bad setting even that was not
enough.

\hypertarget{ex.-2---newton-method}{%
\subsection{Ex. 2 - Newton Method}\label{ex.-2---newton-method}}

For the Newton method we require computing the Hessian \(H(f)^-1\) as
follows: \[
H(f) =
\begin{bmatrix} \frac{\partial^2 f(x,y)}{\partial^2 x} & \frac{\partial^2 f(x,y)}{\partial x \partial y} \\ \frac{\partial^2 f(x,y)}{\partial x \partial y} & \frac{\partial^2 f(x,y)}{\partial^2 y} \end{bmatrix} =
\begin{bmatrix} 1200x^2 + 2 - 400y & -400x \\ -400x & 200 \end{bmatrix}
\]

The determinant of the Hessian is given by: \[
\det (H(f)) = \begin{vmatrix} 1200x^2 + 2 - 400y & -400x \\ -400x & 200 \end{vmatrix}
=200 *(1200x^2+2-400y) - (-400x * -400x) = 400(600x^2+1-200y) - 400^2x^2= \\
400 [600x^2 + 1 - 200y - 400x^2]= 400[200x^2 - 200y + 1]
\] And therefore the inverse is given by: \[
H(f)^{-1}= \frac{1}{400(200x^2-200y+1)}\begin{bmatrix} 200 & 400x \\ 400x & 1200x^2 + 2 - 400y \end{bmatrix}
\]

The whole process of the Newton Method will therefore be as follow, for
some point \(Z^{n}=(x,y)\):

\[
Z^{n+1} = Z^{n} - H(f)^{-1}(Z^{n}) \bullet \nabla f(Z^{n})\\
= \begin{bmatrix} x \\ y \end{bmatrix} - 
\frac{1}{400(200x^2-200y+1)}\begin{bmatrix} 200 & 400x \\ 400x & 1200x^2 + 2 - 400y \end{bmatrix} \begin{bmatrix} 400x^3+2x-2-400xy \\ -200x^2+200y \end{bmatrix} \\
= \begin{bmatrix} x \\ y \end{bmatrix} - 
\frac{1}{100(200x^2-200y+1)}\begin{bmatrix} 100 & 200x \\ 200x & 600x^2 + 1 - 200y \end{bmatrix} \begin{bmatrix} 200x^3+x-1-200xy \\ -100x^2+100y \end{bmatrix} \\
= \begin{bmatrix} x \\ y \end{bmatrix} - 
\frac{1}{100(200x^2-200y+1)}\begin{bmatrix} 100(200x^3+x-1-200xy) + 200x(-100x^2+100y) \\ 200x(200x^3+x-1-200xy) + (600x^2 + 1 - 200y)(-100x^2+100y) \end{bmatrix} \\ 
= \begin{bmatrix} x \\ y \end{bmatrix} - 
\frac{1}{(200x^2-200y+1)} 
\begin{bmatrix} 200x^3+x-1-200xy - 200x^3 +200xy \\ 400x^4+2x^2-2x-400x^2y - 600x^4-x^2+200x^2y+600x^2y+y-200y^2 \end{bmatrix} \\
= \begin{bmatrix} x \\ y \end{bmatrix} - 
\frac{1}{(200x^2-200y+1)} 
\begin{bmatrix} x-1 \\ -200x^4+x^2-2x+400x^2y+y-200y^2 \end{bmatrix} \\
\] So we can implement the following code:

\begin{Shaded}
\begin{Highlighting}[]
\NormalTok{newton.iteration }\OtherTok{\textless{}{-}} \ControlFlowTok{function}\NormalTok{(x,y)\{}
\NormalTok{  mat }\OtherTok{\textless{}{-}} \FunctionTok{as.numeric}\NormalTok{(}\FunctionTok{list}\NormalTok{(x,y)) }\SpecialCharTok{{-}}\NormalTok{ (}\DecValTok{1} \SpecialCharTok{/}\NormalTok{ (}\DecValTok{200} \SpecialCharTok{*}\NormalTok{ x}\SpecialCharTok{\^{}}\DecValTok{2} \SpecialCharTok{{-}} \DecValTok{200} \SpecialCharTok{*}\NormalTok{ y }\SpecialCharTok{+} \DecValTok{1}\NormalTok{)) }\SpecialCharTok{*} \FunctionTok{as.numeric}\NormalTok{(}\FunctionTok{list}\NormalTok{(x}\DecValTok{{-}1}\NormalTok{, }\SpecialCharTok{{-}}\DecValTok{200} \SpecialCharTok{*}\NormalTok{ x}\SpecialCharTok{\^{}}\DecValTok{4} \SpecialCharTok{+}\NormalTok{ x}\SpecialCharTok{\^{}}\DecValTok{2} \SpecialCharTok{{-}} \DecValTok{2} \SpecialCharTok{*}\NormalTok{ x }\SpecialCharTok{+} \DecValTok{400} \SpecialCharTok{*}\NormalTok{ x}\SpecialCharTok{\^{}}\DecValTok{2} \SpecialCharTok{*}\NormalTok{ y }\SpecialCharTok{+}\NormalTok{ y }\SpecialCharTok{{-}} \DecValTok{200} \SpecialCharTok{*}\NormalTok{ y}\SpecialCharTok{\^{}}\DecValTok{2}\NormalTok{))}
  \FunctionTok{return}\NormalTok{(mat)}
\NormalTok{\}}
\end{Highlighting}
\end{Shaded}

Now we can implement the whole Newton method as follows:

\begin{Shaded}
\begin{Highlighting}[]
\NormalTok{newton }\OtherTok{\textless{}{-}} \ControlFlowTok{function}\NormalTok{(a0, tol, maxiter)\{}
  
\NormalTok{  x0 }\OtherTok{\textless{}{-}}\NormalTok{ a0[}\DecValTok{1}\NormalTok{]}
\NormalTok{  x }\OtherTok{\textless{}{-}}\NormalTok{ x0}
\NormalTok{  y0 }\OtherTok{\textless{}{-}}\NormalTok{ a0[}\DecValTok{2}\NormalTok{]}
\NormalTok{  y }\OtherTok{\textless{}{-}}\NormalTok{ y0}
\NormalTok{  a.t }\OtherTok{\textless{}{-}}\NormalTok{ a0}
\NormalTok{  a.values }\OtherTok{\textless{}{-}} \FunctionTok{matrix}\NormalTok{(}\DecValTok{0}\NormalTok{,}\AttributeTok{ncol =} \DecValTok{2}\NormalTok{, }\AttributeTok{nrow =}\NormalTok{ maxiter, }\AttributeTok{byrow =} \ConstantTok{TRUE}\NormalTok{)}
  
\NormalTok{  num.iter }\OtherTok{\textless{}{-}} \DecValTok{0}
  
  \ControlFlowTok{while}\NormalTok{(num.iter }\SpecialCharTok{\textless{}}\NormalTok{ maxiter)\{}
    \CommentTok{\#print("iteration number")}
    \CommentTok{\#print(num.iter)}
    \CommentTok{\#print("x, y values")}
    \CommentTok{\#print(round(x,6))}
    \CommentTok{\#print(round(y,6))}
    \CommentTok{\#print("function value at points")}
    \CommentTok{\#print(round(f(x,y), 6))}
    \CommentTok{\#print("=================")}
    
\NormalTok{    num.iter }\OtherTok{\textless{}{-}}\NormalTok{ num.iter }\SpecialCharTok{+} \DecValTok{1}
\NormalTok{    a.values[num.iter,] }\OtherTok{\textless{}{-}} \FunctionTok{c}\NormalTok{(x,y)}
\NormalTok{    a.t.minus.one }\OtherTok{\textless{}{-}}\NormalTok{ a.t}
    
\NormalTok{    a.t }\OtherTok{\textless{}{-}} \FunctionTok{newton.iteration}\NormalTok{(x,y)}
    
\NormalTok{    x }\OtherTok{\textless{}{-}}\NormalTok{ a.t[[}\DecValTok{1}\NormalTok{]]}
\NormalTok{    y }\OtherTok{\textless{}{-}}\NormalTok{ a.t[[}\DecValTok{2}\NormalTok{]]}
    
    
    \ControlFlowTok{if}\NormalTok{ (}\FunctionTok{norm}\NormalTok{(}\FunctionTok{as.numeric}\NormalTok{(}\FunctionTok{c}\NormalTok{(x,y)) }\SpecialCharTok{{-}} \FunctionTok{as.numeric}\NormalTok{(a.t.minus.one), }\AttributeTok{type =} \StringTok{"2"}\NormalTok{) }\SpecialCharTok{\textless{}}\NormalTok{ tol)\{}
      \ControlFlowTok{break}
\NormalTok{    \}}
    
\NormalTok{  \}}
  \FunctionTok{return}\NormalTok{(}\FunctionTok{list}\NormalTok{(}
\NormalTok{    a.values,}
\NormalTok{    num.iter}
\NormalTok{  ))}
  
\NormalTok{\}}
\end{Highlighting}
\end{Shaded}

\hypertarget{b-plotting-examples}{%
\subsubsection{2b plotting examples}\label{b-plotting-examples}}

First

\hypertarget{first-setting---good-initialization-1}{%
\paragraph{First setting - Good
initialization}\label{first-setting---good-initialization-1}}

Our inital values are \(a^{0}=(0.95, 0.95)\). maxiter is \(1000\).

\begin{Shaded}
\begin{Highlighting}[]
\NormalTok{a0 }\OtherTok{\textless{}{-}} \FunctionTok{c}\NormalTok{(}\FloatTok{0.95}\NormalTok{, }\FloatTok{0.95}\NormalTok{)}
\NormalTok{tol }\OtherTok{\textless{}{-}} \DecValTok{10}\SpecialCharTok{\^{}{-}}\DecValTok{5}
\NormalTok{maxiter }\OtherTok{\textless{}{-}} \DecValTok{1000}

\NormalTok{res }\OtherTok{\textless{}{-}} \FunctionTok{newton}\NormalTok{(a0, tol, maxiter)}
\NormalTok{values }\OtherTok{\textless{}{-}}\NormalTok{ res[[}\DecValTok{1}\NormalTok{]]}
\NormalTok{values[maxiter,] }\OtherTok{\textless{}{-}} \ConstantTok{NA}
\NormalTok{iters }\OtherTok{\textless{}{-}}\NormalTok{ res[[}\DecValTok{2}\NormalTok{]]}
\FunctionTok{print}\NormalTok{(}\FunctionTok{paste}\NormalTok{(}\StringTok{"Iterations required:"}\NormalTok{,iters))}
\end{Highlighting}
\end{Shaded}

\begin{verbatim}
## [1] "Iterations required: 5"
\end{verbatim}

\begin{Shaded}
\begin{Highlighting}[]
\FunctionTok{print}\NormalTok{(}\StringTok{"ZOOM OUT values {-}10 to 10"}\NormalTok{)}
\end{Highlighting}
\end{Shaded}

\begin{verbatim}
## [1] "ZOOM OUT values -10 to 10"
\end{verbatim}

\begin{Shaded}
\begin{Highlighting}[]
\FunctionTok{cf\_func}\NormalTok{(f2, }\AttributeTok{xlim =} \FunctionTok{c}\NormalTok{(}\SpecialCharTok{{-}}\DecValTok{10}\NormalTok{, }\DecValTok{10}\NormalTok{), }\AttributeTok{ylim =} \FunctionTok{c}\NormalTok{(}\SpecialCharTok{{-}}\DecValTok{10}\NormalTok{, }\DecValTok{10}\NormalTok{), }\AttributeTok{bar =} \ConstantTok{TRUE}\NormalTok{, }\AttributeTok{pts =}\NormalTok{ values)}
\end{Highlighting}
\end{Shaded}

\includegraphics{ex2_files/figure-latex/graph1-newton-1.pdf}

\begin{Shaded}
\begin{Highlighting}[]
\FunctionTok{print}\NormalTok{(}\StringTok{"ZOOM OUT values {-}2.5 to 2.5"}\NormalTok{)}
\end{Highlighting}
\end{Shaded}

\begin{verbatim}
## [1] "ZOOM OUT values -2.5 to 2.5"
\end{verbatim}

\begin{Shaded}
\begin{Highlighting}[]
\FunctionTok{cf\_func}\NormalTok{(f2, }\AttributeTok{xlim =} \FunctionTok{c}\NormalTok{(}\SpecialCharTok{{-}}\FloatTok{2.5}\NormalTok{, }\FloatTok{2.5}\NormalTok{), }\AttributeTok{ylim =} \FunctionTok{c}\NormalTok{(}\SpecialCharTok{{-}}\FloatTok{2.5}\NormalTok{, }\FloatTok{2.5}\NormalTok{), }\AttributeTok{bar =} \ConstantTok{TRUE}\NormalTok{, }\AttributeTok{pts =}\NormalTok{ values)}
\end{Highlighting}
\end{Shaded}

\includegraphics{ex2_files/figure-latex/graph1-newton-2.pdf}

\begin{Shaded}
\begin{Highlighting}[]
\FunctionTok{print}\NormalTok{(}\StringTok{"ZOOM IN values 0.8{-}1.2"}\NormalTok{)}
\end{Highlighting}
\end{Shaded}

\begin{verbatim}
## [1] "ZOOM IN values 0.8-1.2"
\end{verbatim}

\begin{Shaded}
\begin{Highlighting}[]
\FunctionTok{cf\_func}\NormalTok{(f2, }\AttributeTok{xlim =} \FunctionTok{c}\NormalTok{(}\FloatTok{0.8}\NormalTok{, }\FloatTok{1.2}\NormalTok{), }\AttributeTok{ylim =} \FunctionTok{c}\NormalTok{(}\FloatTok{0.8}\NormalTok{, }\FloatTok{1.2}\NormalTok{), }\AttributeTok{bar =} \ConstantTok{TRUE}\NormalTok{, }\AttributeTok{pts =}\NormalTok{ values)}
\end{Highlighting}
\end{Shaded}

\includegraphics{ex2_files/figure-latex/graph1-newton-3.pdf}

\begin{Shaded}
\begin{Highlighting}[]
\FunctionTok{print}\NormalTok{(}\StringTok{"ZOOM More IN values 0.975{-}1.025"}\NormalTok{)}
\end{Highlighting}
\end{Shaded}

\begin{verbatim}
## [1] "ZOOM More IN values 0.975-1.025"
\end{verbatim}

\begin{Shaded}
\begin{Highlighting}[]
\FunctionTok{cf\_func}\NormalTok{(f2, }\AttributeTok{xlim =} \FunctionTok{c}\NormalTok{(}\FloatTok{0.975}\NormalTok{, }\FloatTok{1.025}\NormalTok{), }\AttributeTok{ylim =} \FunctionTok{c}\NormalTok{(}\FloatTok{0.975}\NormalTok{, }\FloatTok{1.025}\NormalTok{), }\AttributeTok{bar =} \ConstantTok{TRUE}\NormalTok{, }\AttributeTok{pts =}\NormalTok{ values)}
\end{Highlighting}
\end{Shaded}

\includegraphics{ex2_files/figure-latex/graph1-newton-4.pdf} \#\#\#\#
Second setting - average initialization Our inital values are
\(a^{0}=(5, 5)\). maxiter is \(10000\).

\begin{Shaded}
\begin{Highlighting}[]
\NormalTok{a0 }\OtherTok{\textless{}{-}} \FunctionTok{c}\NormalTok{(}\DecValTok{5}\NormalTok{,}\DecValTok{5}\NormalTok{)}
\NormalTok{tol }\OtherTok{\textless{}{-}} \DecValTok{10}\SpecialCharTok{\^{}{-}}\DecValTok{5}
\NormalTok{maxiter }\OtherTok{\textless{}{-}} \DecValTok{1000}

\NormalTok{res }\OtherTok{\textless{}{-}} \FunctionTok{newton}\NormalTok{(a0, tol, maxiter)}
\NormalTok{values }\OtherTok{\textless{}{-}}\NormalTok{ res[[}\DecValTok{1}\NormalTok{]]}
\NormalTok{values[maxiter,] }\OtherTok{\textless{}{-}} \ConstantTok{NA}
\NormalTok{iters }\OtherTok{\textless{}{-}}\NormalTok{ res[[}\DecValTok{2}\NormalTok{]]}
\FunctionTok{print}\NormalTok{(}\FunctionTok{paste}\NormalTok{(}\StringTok{"Iterations required:"}\NormalTok{,iters))}
\end{Highlighting}
\end{Shaded}

\begin{verbatim}
## [1] "Iterations required: 5"
\end{verbatim}

\begin{Shaded}
\begin{Highlighting}[]
\FunctionTok{print}\NormalTok{(}\StringTok{"ZOOM OUT values {-}10 to 10"}\NormalTok{)}
\end{Highlighting}
\end{Shaded}

\begin{verbatim}
## [1] "ZOOM OUT values -10 to 10"
\end{verbatim}

\begin{Shaded}
\begin{Highlighting}[]
\FunctionTok{cf\_func}\NormalTok{(f2, }\AttributeTok{xlim =} \FunctionTok{c}\NormalTok{(}\SpecialCharTok{{-}}\DecValTok{10}\NormalTok{, }\DecValTok{10}\NormalTok{), }\AttributeTok{ylim =} \FunctionTok{c}\NormalTok{(}\SpecialCharTok{{-}}\DecValTok{10}\NormalTok{, }\DecValTok{10}\NormalTok{), }\AttributeTok{bar =} \ConstantTok{TRUE}\NormalTok{, }\AttributeTok{pts =}\NormalTok{ values)}
\end{Highlighting}
\end{Shaded}

\includegraphics{ex2_files/figure-latex/graph2-newton-1.pdf}

\begin{Shaded}
\begin{Highlighting}[]
\FunctionTok{print}\NormalTok{(}\StringTok{"ZOOM OUT values {-}2.5 to 2.5"}\NormalTok{)}
\end{Highlighting}
\end{Shaded}

\begin{verbatim}
## [1] "ZOOM OUT values -2.5 to 2.5"
\end{verbatim}

\begin{Shaded}
\begin{Highlighting}[]
\FunctionTok{cf\_func}\NormalTok{(f2, }\AttributeTok{xlim =} \FunctionTok{c}\NormalTok{(}\SpecialCharTok{{-}}\FloatTok{2.5}\NormalTok{, }\FloatTok{2.5}\NormalTok{), }\AttributeTok{ylim =} \FunctionTok{c}\NormalTok{(}\SpecialCharTok{{-}}\FloatTok{2.5}\NormalTok{, }\FloatTok{2.5}\NormalTok{), }\AttributeTok{bar =} \ConstantTok{TRUE}\NormalTok{, }\AttributeTok{pts =}\NormalTok{ values)}
\end{Highlighting}
\end{Shaded}

\includegraphics{ex2_files/figure-latex/graph2-newton-2.pdf}

\begin{Shaded}
\begin{Highlighting}[]
\FunctionTok{print}\NormalTok{(}\StringTok{"ZOOM IN values 0.8{-}1.2"}\NormalTok{)}
\end{Highlighting}
\end{Shaded}

\begin{verbatim}
## [1] "ZOOM IN values 0.8-1.2"
\end{verbatim}

\begin{Shaded}
\begin{Highlighting}[]
\FunctionTok{cf\_func}\NormalTok{(f2, }\AttributeTok{xlim =} \FunctionTok{c}\NormalTok{(}\FloatTok{0.8}\NormalTok{, }\FloatTok{1.2}\NormalTok{), }\AttributeTok{ylim =} \FunctionTok{c}\NormalTok{(}\FloatTok{0.8}\NormalTok{, }\FloatTok{1.2}\NormalTok{), }\AttributeTok{bar =} \ConstantTok{TRUE}\NormalTok{, }\AttributeTok{pts =}\NormalTok{ values)}
\end{Highlighting}
\end{Shaded}

\includegraphics{ex2_files/figure-latex/graph2-newton-3.pdf}

\begin{Shaded}
\begin{Highlighting}[]
\FunctionTok{print}\NormalTok{(}\StringTok{"ZOOM More IN values 0.975{-}1.025"}\NormalTok{)}
\end{Highlighting}
\end{Shaded}

\begin{verbatim}
## [1] "ZOOM More IN values 0.975-1.025"
\end{verbatim}

\begin{Shaded}
\begin{Highlighting}[]
\FunctionTok{cf\_func}\NormalTok{(f2, }\AttributeTok{xlim =} \FunctionTok{c}\NormalTok{(}\FloatTok{0.975}\NormalTok{, }\FloatTok{1.025}\NormalTok{), }\AttributeTok{ylim =} \FunctionTok{c}\NormalTok{(}\FloatTok{0.975}\NormalTok{, }\FloatTok{1.025}\NormalTok{), }\AttributeTok{bar =} \ConstantTok{TRUE}\NormalTok{, }\AttributeTok{pts =}\NormalTok{ values)}
\end{Highlighting}
\end{Shaded}

\includegraphics{ex2_files/figure-latex/graph2-newton-4.pdf} \#\#\#\#
Third setting - bad initialization Our inital values are
\(a^{0}=(-10, 10)\). maxiter is \(1000\).

\begin{Shaded}
\begin{Highlighting}[]
\NormalTok{a0 }\OtherTok{\textless{}{-}} \FunctionTok{c}\NormalTok{(}\SpecialCharTok{{-}}\DecValTok{10}\NormalTok{,}\DecValTok{10}\NormalTok{)}
\NormalTok{tol }\OtherTok{\textless{}{-}} \DecValTok{10}\SpecialCharTok{\^{}{-}}\DecValTok{5}
\NormalTok{maxiter }\OtherTok{\textless{}{-}} \DecValTok{1000}

\NormalTok{res }\OtherTok{\textless{}{-}} \FunctionTok{newton}\NormalTok{(a0, tol, maxiter)}
\NormalTok{values }\OtherTok{\textless{}{-}}\NormalTok{ res[[}\DecValTok{1}\NormalTok{]]}
\NormalTok{values[maxiter,] }\OtherTok{\textless{}{-}} \ConstantTok{NA}
\NormalTok{iters }\OtherTok{\textless{}{-}}\NormalTok{ res[[}\DecValTok{2}\NormalTok{]]}
\FunctionTok{print}\NormalTok{(}\FunctionTok{paste}\NormalTok{(}\StringTok{"Iterations required:"}\NormalTok{,iters))}
\end{Highlighting}
\end{Shaded}

\begin{verbatim}
## [1] "Iterations required: 5"
\end{verbatim}

\begin{Shaded}
\begin{Highlighting}[]
\FunctionTok{print}\NormalTok{(}\StringTok{"ZOOM OUT values {-}10 to 10"}\NormalTok{)}
\end{Highlighting}
\end{Shaded}

\begin{verbatim}
## [1] "ZOOM OUT values -10 to 10"
\end{verbatim}

\begin{Shaded}
\begin{Highlighting}[]
\FunctionTok{cf\_func}\NormalTok{(f2, }\AttributeTok{xlim =} \FunctionTok{c}\NormalTok{(}\SpecialCharTok{{-}}\DecValTok{10}\NormalTok{, }\DecValTok{10}\NormalTok{), }\AttributeTok{ylim =} \FunctionTok{c}\NormalTok{(}\SpecialCharTok{{-}}\DecValTok{10}\NormalTok{, }\DecValTok{10}\NormalTok{), }\AttributeTok{bar =} \ConstantTok{TRUE}\NormalTok{, }\AttributeTok{pts =}\NormalTok{ values)}
\end{Highlighting}
\end{Shaded}

\includegraphics{ex2_files/figure-latex/graph3-newton-1.pdf}

\begin{Shaded}
\begin{Highlighting}[]
\FunctionTok{print}\NormalTok{(}\StringTok{"ZOOM OUT values {-}2.5 to 2.5"}\NormalTok{)}
\end{Highlighting}
\end{Shaded}

\begin{verbatim}
## [1] "ZOOM OUT values -2.5 to 2.5"
\end{verbatim}

\begin{Shaded}
\begin{Highlighting}[]
\FunctionTok{cf\_func}\NormalTok{(f2, }\AttributeTok{xlim =} \FunctionTok{c}\NormalTok{(}\SpecialCharTok{{-}}\FloatTok{2.5}\NormalTok{, }\FloatTok{2.5}\NormalTok{), }\AttributeTok{ylim =} \FunctionTok{c}\NormalTok{(}\SpecialCharTok{{-}}\FloatTok{2.5}\NormalTok{, }\FloatTok{2.5}\NormalTok{), }\AttributeTok{bar =} \ConstantTok{TRUE}\NormalTok{, }\AttributeTok{pts =}\NormalTok{ values)}
\end{Highlighting}
\end{Shaded}

\includegraphics{ex2_files/figure-latex/graph3-newton-2.pdf}

\begin{Shaded}
\begin{Highlighting}[]
\FunctionTok{print}\NormalTok{(}\StringTok{"ZOOM IN values 0.8{-}1.2"}\NormalTok{)}
\end{Highlighting}
\end{Shaded}

\begin{verbatim}
## [1] "ZOOM IN values 0.8-1.2"
\end{verbatim}

\begin{Shaded}
\begin{Highlighting}[]
\FunctionTok{cf\_func}\NormalTok{(f2, }\AttributeTok{xlim =} \FunctionTok{c}\NormalTok{(}\FloatTok{0.8}\NormalTok{, }\FloatTok{1.2}\NormalTok{), }\AttributeTok{ylim =} \FunctionTok{c}\NormalTok{(}\FloatTok{0.8}\NormalTok{, }\FloatTok{1.2}\NormalTok{), }\AttributeTok{bar =} \ConstantTok{TRUE}\NormalTok{, }\AttributeTok{pts =}\NormalTok{ values)}
\end{Highlighting}
\end{Shaded}

\includegraphics{ex2_files/figure-latex/graph3-newton-3.pdf}

\begin{Shaded}
\begin{Highlighting}[]
\FunctionTok{print}\NormalTok{(}\StringTok{"ZOOM More IN values 0.975{-}1.025"}\NormalTok{)}
\end{Highlighting}
\end{Shaded}

\begin{verbatim}
## [1] "ZOOM More IN values 0.975-1.025"
\end{verbatim}

\begin{Shaded}
\begin{Highlighting}[]
\FunctionTok{cf\_func}\NormalTok{(f2, }\AttributeTok{xlim =} \FunctionTok{c}\NormalTok{(}\FloatTok{0.975}\NormalTok{, }\FloatTok{1.025}\NormalTok{), }\AttributeTok{ylim =} \FunctionTok{c}\NormalTok{(}\FloatTok{0.975}\NormalTok{, }\FloatTok{1.025}\NormalTok{), }\AttributeTok{bar =} \ConstantTok{TRUE}\NormalTok{, }\AttributeTok{pts =}\NormalTok{ values)}
\end{Highlighting}
\end{Shaded}

\includegraphics{ex2_files/figure-latex/graph3-newton-4.pdf}

\hypertarget{c}{%
\subsubsection{1c}\label{c}}

We can see that the convergence is much faster with Newton process, even
with a higher tolerance the convergence happens faster and no more than
6 iterations were required.

\end{document}
